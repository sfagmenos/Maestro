\section{Introduction}
\label{sect:intro}
The \lang{} language will help users for executing a large amount of scripts into a pool of workers.
Here we will give some program examples on how to create jobs, define dependencies between them and
explain all the feautures \lang{} is offering.


The main functionality will be demonstrated in the examples that will follow. Starting from a 
proper "Hello World!" program will continue with more complex ones in order to give fare examples
for all the uses of the language.


With the arrival of the virtual machines most of the modern data centres switch from mainframes to
virtual machines that are spawm when the data centre needs more computing power. More and more companies,
institutions and large expreiments around the world are looking into the agile way of doing computing. From the
pioneer Amazon with the cloud infastructure to the FOSS reliable solution of Openstack from RedHat,
data centres around the world are changing and with them changes the computing itself.
As the user can be unaware of the spawning, \lang{} will come to fill such gap. Either the user will
introduce a pool of workers or a cluster of machines(like clic.cs.columbia.edu), \lang{} will be 
able to distrubute the jobs in the less loaded machine for faster completion. Having also a REPL
interface will help the user to experiment with \lang{} before starts to create more complex 
programs. He can also play with the examples we are having in this tutorial without breaking a sweat.
