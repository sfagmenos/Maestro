\section{Introduction}
\label{sect:intro}

%In the era of virtualization modern data centres switch from traditional servers to virtual machines.
%
%Companies, institutions, and large scientific organizations are now following an agile way of computing.
%mazon with the cloud infastructure to the FOSS reliable solution of Openstack from RedHat,
%data centres around the world are changing and with them changes the computing itself.
%As the user can be unaware of the spawning, \lang{} will come to fill such gap. Either the user will
%introduce a pool of workers or a cluster of machines(like clic.cs.columbia.edu), \lang{} will be 
%able to distrubute the jobs in the less loaded machine for faster completion. Having also a REPL
%interface will help the user to experiment with \lang{} before starts to create more complex 
%p
%
Frequently a user allocates a pool of machines in order to conduct experimental
evaluation represented in the format of scripts. In a broad sense, an end-user
operates on a pool of machines and allocates jobs represented in a scripting
language. \lang{} is a programming language which provides user with ability
to express powerful semantics related to job distribution and scheduling.
The purpose of this tutorial is to demonstrate features and functionality 
supported by  \lang{}. Also, we provide exemplarey programs  to
explain how \lang{}  should be used to create jobs, describe dependencies amongst jobs,
and execute jobs on a cluster of workers. 



The main functionality will be demonstrated in the examples that will follow. Starting from a 
proper "Hello World!" program will continue with more complex ones in order to give fare examples
for all the aspects of the language.


rograms. He can also play with the examples we are having in this tutorial without breaking a sweat.
