\documentclass[12pt]{article}
\usepackage{amssymb,amsmath,amsthm}
\usepackage[pdftex]{graphicx}
\usepackage{tikz}
\usepackage{ulem}
\usepackage{float}
\usepackage{hyperref}
\usepackage[top=1in, bottom=1in, left=.5in, right=1in]{geometry}
\usepackage{algorithm}
\usepackage{algpseudocode} 
\usepackage{bbm}
\usepackage{textcomp}
\usepackage{lmodern}
\usepackage[T1]{fontenc}
\usepackage[latin9]{inputenc}
\usepackage{geometry}
\geometry{verbose,tmargin=2cm,bmargin=3cm,lmargin=2cm,rmargin=2cm,headheight=2cm,headsep=2cm,footskip=2cm}
\usepackage{amstext}
% \usepackage[margin=1in]{geometry}
\linespread{1.5}

\title{Language Reference Manual Operators}

\author{Mathias Lécuyer, Vaggelis Atlidakis, Georgios Koloventzos, Yiren Lu}
\date{\today}

\begin{document}

\maketitle
\tableofcontents

\subsection{Unary arithmetic operations}

$$unaryExpr \rightarrow ``-'' unaryExpr$$

The minus symbol $``-''$ returns the negation of the argument, which needs to be numeric. If the argument has the wrong time, the error \textbf{TypeError} is returned.

\subsection{Binary arithmetic operations}

There are two priority levels for binary arithmetic operators, one for addition/subtraction and another for multiplication/division.

$$multExpr \rightarrow multExpr ``*'' unaryExpr$$
$$ \vert multExpr ``/'' unaryExpr$$
$$ \vert multExpr ``\%'' unaryExpr$$
$$ \vert unaryExpr$$

$$addExpr \rightarrow addExpr ``+'' unaryExpr $$ 
$$\vert addExpr ``-'' unaryExpr$$
$$\vert multExpr$$

The multiplication symbol $``*''$ gives the product of the two operands, which both must be numbers (ie either integer or floats), and are first converted to common type.

The division symbol $``/''$ gives the quotient of the two operands, both of which must be numeric. Division by zero raises the \textbf{DivideByZero} error.

The addition symbol $``+''$ gives the sum of the two operands, both of which must be numeric.

The subtraction symbol $``-''$ gives the difference of the two operands, both of which must be numeric.

The modulo symbol $``\%''$ gives the remainder of the quotient of the two operands. A right argument with value zero will raise the \textbf{DivideByZero} error.

\subsection{Comparison Operators}

Comparisons have a lower priority than the arithmetic operations

$$comp \rightarrow addExpr (compOperator \text{ } addExpr)*$$
$$compOperator \rightarrow ``<'' $$
$$\vert ``>'' $$
$$\vert ``=='' $$
$$\vert ``<='' $$
$$\vert ``>='' $$
$$\vert ``!=''$$

The results of comparisons are boolean values, true and false.

Comparisons can be chained and are equivalent to their constituent pieces, i.e. $x < y < z$ is equivalent to $x < y$ and $y < z$, although evaluation occurs right to left, up to correctness.

\subsection{Dependencies}

$$dep \rightarrow addExpr (depOperator \text{ } addExpr)*$$
$$depOperator \rightarrow ``->'' \vert ``<->''$$

The results of dependencies are boolean values, true and false. 

\subsection{Priorities}

$$priorities \rightarrow addExpr (priorOperator \text{ } addExpr)*$$
$$priorOperator \rightarrow ``~>'' \vert ``<~>''$$

The results of dependencies are boolean values, true and false.










\section{Introduction}
The Maestro language is a declarative language to schedule jobs with hard and soft dependencies, locally or over the network.
% The C Flat language is mostly a subset of the C language. Some of the core functionalities of C has been stripped:
% there is no preprocessor, no structs, no strings, not even pointers. it’s goal is purely educational. Originally Nico
% and Dan were working on two separate languages. The two projects merged, taking some features from each, and
% this is the resulting language. This document is inspired by the C Reference Manuel by Dennis Ritchie.

\section{Lexical conventions}
\subsection{Whitespace}
Whitespaces are either a tab or a white space. At least one white space is required to separate identifiers from
each other, and certain operators.

\subsection{Comments}
Comments are denoted by // and represent a comment that ends with the end of the current line.

\subsection{Identifiers}
An identifier is a sequence of letters, digits, and underscores. The first character must not be a digit.

\subsection{Keywords}
The following identifiers are used as keywords, and are reserved names:\\
if else map reduce run Job

\subsection{Built in functions}
The following functions are built into the language, and their names are reserved:\\
map reduce run Job\\
All return a list of Jobs.
\begin{description}
  \item[Job(string, string..)] Job takes a string that is the path of a file to
    execute for the job, and any number of strings that will be passed as
    arguments. It returns a list with a Job object.
  \item[map(list, string)] the first list is a list of jobs that will run first, and their
    stdout will be concatenated. For every line in the concatenated stdout, map will create
    a Job with the string argmuent and the stdout line as an argument. It returns the list
    of created Jobs. This call is asynchronous.
  \item[reduce(list, string)] the first list is a list of jobs that will run first (usually the
    output of a map). String is the path to the reduce program. There will be one Job
    created with that program, with arguments that are the output from the previous reduce and
    the output of the next Job from the list. The first reduce will have nil as first argument.
  \item[run(list)] run takes an arbitrary number of comma separated lists of Jobs
    and runs them. If they have dependencies, these dependencies will run first.
\end{description}

\subsection{Reserved}
The following characters are reserved: $\texttt{+ - / * \% -> => <-> \textasciitilde>
\textasciitilde< \textasciitilde , ( ) " .}$.

\subsection{Types}
Maestro is dynamically typed and doesn't allow users to create new types.
\subsubsection{Basic Types}
\begin{description}
  \item[Job] A Job represents a program that can be ran on a worker. It encapsulates
    the path of the program and exposes outputs and errors.
  \item[Int] Ints are signed integers with arbitrary size.
  \item[String] Strings are delimited by " and have to be on one line.
  \item[Boolean] Boolean encapsulates true or false.
\end{description}

\subsubsection{Derived Types}
\begin{description}
  \item[List] Lists are ordered collections of one or more values. These
    objects do not need to have the same type. The literal representation is a
    comma separated sequence of expressions surrounded by brackets ([ and ]).
\end{description}

\subsection{Scope}
Maestro is a scripting language to quickly distributed interdependent jobs. Its
intended use is for scripts that glue tasks together. The scope of a variable
is thus the entire script, or the whole lifetime of the interpreter.

\subsection{Operators}
Operators are left associative and have the following precedences (same line
means same precedence, lower line means higher precedence):\\
$\texttt{+ - -> => \textasciitilde> \textasciitilde<}$\\
$\texttt{/ * <-> \textasciitilde}$\\
$\texttt{\%}$.

\subsection{Expressions}
\subsubsection{identifier}
An identifier evaluates to the value of the corresponding variable.

\subsubsection{( expression )}
A parenthesized expression takes the value of the evaluated expression.

\subsubsection{identifier ( expression-listopt )}
A function call is an expression. The arguments are comma separated and are evaluated
from left to right, before the call. It is possible to have 0 arguments.

\subsubsection{expression + expression}
The result is the sum of the expressions if they evaluate to an Int.

\subsubsection{expression - expression}
The result is the difference of the expressions if they evaluate to an Int.

\subsubsection{expression * expression}
The result is the multiplication of the expressions if they evaluate to an Int.

\subsubsection{expression / expression}
The result is the Integer division of the expressions if they evaluate to an Int.

\subsubsection{expression \% expression}
The \% operator yields the remainder from the division of the first expression by the second, if
both expressions evaluate to an Int.

\subsubsection{expression -> expression}
\subsubsection{expression <-> expression}
% The value of the right hand side operand should be non-negative and less than 32, if not the result is undefined.
% The value of “E1 >> E2” is E1 arithmetically right-shifted by E2 bit positions. Vacated bits are filled by a copy of
% the sign bit of the first expression.
% The value of “E1 << E2” is R1 left-shifted by E2 bit positions. Vacated bits are 0-filled.
% 3.19
% expression
% expression
% expression
% expression
% < expression
% > expression
% <= expression
% >= expression
% The operators < (less than), > (greater than), <= (less than or equal to), >= (greater than or equal to) all yield 0 if
% the specified relation is false and 1 if it is true.
% 3.20
\subsubsection{expression == expression}
\subsubsection{expression != expression}
% The operators == (equal to) and the != (not equal to) yield 0 if the specified relation is false, 1 if it is true.
% 3.21
% expression & expression
% The & operator yield the bitwise and function of the operands.
% 3.22
% expression ˆ expression
% The & operator yield the bitwise exclusive or function of the operands.
% 3.23
% expression | expression
% The | operator yield the bitwise inclusive or function of the operands.
% 3.24
% expression && expression
% The && operator returns 1 if both operands are non-zero, 0 otherwise. Both operands are always evaluated.
% 3.25
% expression || expression
% The || operator returns 1 if either of its operands is non-zero, 0 otherwise. Both operands are always evaluated.
% 3
% 3.26
% identifier = expression
% The value of the referred variable is replaced by the value of the expression.
% 3.27
% identifier
% identifier
% identifier
% identifier
% identifier
% identifier
% identifier
% identifier
% identifier
% identifier
% += expression
% -= expression
% *= expression
% /= expression
% %= expression
% >>= expression
% <<= expression
% &= expression
% ˆ= expression
% |= expression
% An expression of the form “id op= expr” is equivalent to “id = id op expr”.
% 4
% Statements
% Statements are executed in sequence.
% 4.1
% Expression statement
% Most statement are expression statements, which have the form
% expression ;
% 4.2
% Compound statement
% So that several statements can be used where one is expected, the compound statement is provided:
% compound-statement:
% { statement-listopt }
% statement-list:
% statement
% statement statement-list
% 4.3
% Conditional statement
% The two forms of the conditional statement are
% if ( expression ) statement
% if ( expression ) statement else statement
% In both cases the expression is evaluated and if it is non-zero, the first substatement is executed. In the second case
% the second substatement is executed if the expression is 0. As usual the “else” ambiguity is resolved by connecting
% an else with the last encountered elseless if.
% 4
% 4.4
% While statement
% The while statement has the form
% while ( expression ) statement
% The substatement is executed repeatedly so long as the value of the expression remains non-zero. The test takes
% place before each execution of the statement.
% 4.5
% For statement
% The for statement has the form
% for ( expression-1opt ; expression-2opt ; expression-3opt ) statement
% This statement is equivalent to
% expression-1;
% while ( expression-2 ) {
% statement
% expression-3;
% }
% Any or all the expression may be dropped. A missing expression-2 makes the implied while clause equivalent
% to “while(1)”. Other missing expressions are simply dropped from the expansion above.
% 4.6
% Break statement
% The statement
% break;
% casuses termination of the smallest enclosing while or for statement; control passes to the statement following the
% terminated statement.
% 4.7
% Continue statement
% The statement
% continue;
% causes control to pass to the loop-continuation portion of the smallest enclosing while or for statement; that is to
% the end of the loop. In case of a for loop of the form “for(e1;e2;e3) {...}”, e3 is evaluated before checking e2.
% 4.8
% Return statement
% A function returns to its caller by means of the return statement
% return expression ;
% The value of the expression is returned to the caller of the function.
% 4.9
% Null statement
% The null statement has the form
% ;
% A null statement is useful to supply a null body to a looping statement such as while.
% 5
% 4.10
% Try-catch statement
% The two form of the try-catch statement are
% try { statement-listopt }catch ( identifier ) { statement-listopt }
% try { statement-listopt }catch { statement-listopt }
% The statments enclosed in the try block are executed until an exception is thrown. In case no exception is thrown,
% the statments enclosed in the catch block are not executed. The first form of the try-catch statement allows to
% assign the value of the exception to a variable. Try-catch statement dynamically nest across function calls.
% 4.11
% Throw statement
% The throw statement has the form
% throw expression ;
% Throwing an exception causes control to pass to the catch block of the nearest dynamically-enclosing try-catch
% statement. If none is found, it causes the program to terminate with an error. The given expression is the value of
% the thrown exception.
% 5
% Program definition
% A ltc program consists of a sequence of function definition.
% program:
% function-definition
% function-definition program
% function-definition:
% identifier ( parameter-listopt ) { statement-listopt }
% parameter-list:
% identifier
% identifier , parameter-list
% the same identifier cannot be used more than once in the parameter list. Within the same program, A function
% cannot be defined twice (name wise).
% All functions return a integer value. A function can return to the caller without an explicit return statement, in
% this case the return value is undefined.
% A simple example of a complete function definition:
% max (a, b, c) {
% if (a > b) m = a; else m = b;
% if (m > c) return m; else return c;
% }
% 6
% 6
% Scope rules
% There are no global variables, but only local variables which are statically binded. The scope of a local variable is
% the whole function where the variable is used. The scope of function parameters is the whole function.
% Function scope is the entire program.
% 7
% Declarations
% Variables don’t need to be declared, they are initialized to 0.
% A function call can be made whether or not the function actually exists, the program will simply not link if a call
% to a non-existing function is made.
% 8
% Namespace
% Variables and function use different namespaces. Therefore such a function is correct: “f() {f=1; return f; }”.
% 9
% 9.1
% Syntax Summary
% Expressions
% expression:
% identifier
% literal
% ( expression )
% identifier ( expression-listopt )
% -expression
% +expression
% !expression
% ∼expression
% ++identifier
% --identifier
% identifier++
% identifier--
% expression binop expression
% identifier asgnop expression
% expression-list:
% expression
% expression , expression-list
% The unary operators - + ! ∼ have higher priority than binary operator.
% Binary operators all group left to right and have priority decreasing as indicated:
% binop:
% * / %
% 7
% + -
% >> <<
% < > <= >=
% == !=
% &
% ∧
% |
% &&
% ||
% Assignment operator all have the same priority, and all group right to left.
% asgnop:
% = += -= *= /= %= >>= <<= &= ∧ = |=
% 9.2
% Statements
% statement:
% expression ;
% { statement-listopt }
% if ( expression ) statement
% if ( expression ) statement else statement
% while ( expression ) statement
% for ( expressionopt ; expressionopt ; expressionopt ) statement
% break;
% continue;
% return expression;
% try { statement-listopt }catch { statement-listopt }
% try { statement-listopt }catch ( identifier ) { statement-listopt }
% throw expression;
% ;
% statement-list:
% statement
% statement statement-list
% 8
% 9.3
% Program definition
% program:
% function-definition
% function-definition program
% function-definition:
% identifier ( parameter-listopt ) { statement-listopt }
% parameter-list:
% identifier
% identifier , parameter-list
% 9































\end{document}

