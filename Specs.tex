\section{Language Specifications and Main Features}
\label{sect:spec}

\subsection*{Operators}
In what follows we present tables demonstrating the operators supported by our
language.%, along with some indicative example.


\begin{table}[h]
\begin{center}
    \parbox{.45\linewidth}{
        \begin{tabular}{| l | l |}
        \hline
        Operator & $a=4, b=2$ \\ \hline
        $+$ &  $a + b$ gives 6 \\  \hline
        $-$ &  $a - b$ gives 2  \\ \hline
        $/$ &  $a~/~b$ gives gives 2 \\ \hline
        $*$ &  $a~*~b$ gives 8  \\ \hline
        $\%$ & $a~\%~b$ gives 0 \\ \hline
        \end{tabular}
        \caption{Arithmetic operators.}
    }
    \parbox{.45\linewidth}{
        \begin{tabular}{| l | l |}
        \hline
        Operator & $a=4, b=2$ \\ \hline
        $=$  &  $a = b$  $a$ has taken on the value of 2 \\ \hline
        $+=$ &  $a += b$ is equivalent to a = a + b\\ \hline
        $-=$ &  $a -= b$ is equivalent to a = a - b \\ \hline
        $+=$ &  $a *= b$ is equivalent to a = a * b\\ \hline
        $/=$ &  $a /= b$ is equivalent to a = a / b\\ \hline
        \end{tabular}
        \caption{Assignment operators.}
    }
\end{center}
\end{table}

\begin{table}[h]
\begin{center}
    \parbox{.4\linewidth}{
        \begin{tabular}{| l | l |}
        \hline
        Operator & $a=4, b=2$ \\ \hline
        $==$ & $a == b$ is not true \\ \hline
        $!=$ & $a != b$ is true \\ \hline
        $\textless$  & $a \textless b$ is false \\ \hline
        $\textgreater$  & $a \textgreater b$ is true \\ \hline
        $\textless=$ & $a \textless= b$ is not true \\ \hline
        $\textgreater=$ & $a \textgreater= b$ is true \\ \hline
        \end{tabular}
        \caption{Comparison operators.}
    }
    \parbox{.4\linewidth}{
        \begin{tabular}{| l | l |}
        \hline
        Operator & $a = true, b = false$ \\ \hline
        and & a and b is false \\ \hline
        or & a or b is true\\ \hline
        not & not b is true \\ \hline
        in & a in b is true \\ \hline
        not in & a not in b is false\\ \hline
        \end{tabular}
        \caption{Logical operators}
     }
\end{center}
\end{table}

The order of supported operations
is: (1) Exponentiation, (2) Multiply, Divide, Modulo, (3) Addition, Subtraction,
(4) Comparison, (5) Assignment, (6) Membership, and (7) Logical.

\subsection*{Datatypes}
Variables have to be declared before use, and datatypes of
variables are specified at declaration time. Variable names 
begin with an alphabet letter.

\begin{description}
\item [Job:] The Job type is a \lang{} specific type, equivalent to a struct
with fields describing a Job to be run. The fields are: 'script' for the
name of the script implementing the job, 'workers' for the number of workers to use
(default is as many as possible), and 'priority', an Int from 0 to 5, 0 being the
highest priority.
Job variables are declared as:
$Job~\textless var\textgreater = Job.new(~script=<String>,~workers=<Int>,~priority=<Int>)$

\item [Int:] Integer variables can store integer values like 0, 7, -1024,
and support both signed and unsigned values. Integer variables are declared
using the 'Int' keyword as:
$Int~\textless var\textgreater,~\textless var\textgreater, ...$
%$Int \textlessvar\textgreater = \textlessvalue\textgreater$

%$Int a, b, c$
%$Int a = 234$

\item [Float:] Floating point datatypes can store real values like 1.03 or -23.56.
Floating Point variables are declared using the 'Float' keyword as:
$Float~\textless  var\textgreater,~\textless  var\textgreater, ...$
%$Float \textlessvar\textgreater = \textlessvalue\textgreater$

%Float a, b, c
%Float a = 23.78


\item [Bool:] Boolean datatypes can only hold one of two possible values:
True or False. Boolean variables are declared using the 'Bool' keyword as
$Bool~\textless  var\textgreater,~\textless  var\textgreater, ...$
%$Bool \textlessvar\textgreater = [\textlessTrue\textgreater OR \textlessFalse\textgreater]$
%
%$Bool a, b, c$
%$Bool a = True$



\item [String:] String datatypes represent a sequence of characters. String
variables are declared using the 'String' keyword, and string values are 
declared within single quotation marks.
String variables are declared as:
$String~\textless var\textgreater,~\textless  var\textgreater, ...$
%$String \textlessvar\textgreater = ' \textlessvalue\textgreater '$

%$String a, b, c$
%$String a = 'test'$

\item [List:]
List type is a container that holds a number of other objects of the same type,
in a given order. The list type implements a sequence, and also allows user to
add/remove objects from the sequence.

Lists are created as:
$L = type[\textless expr \textgreater,~\textless expr \textgreater, ...]$
, e.g., $L = int[23,~45,~2]$.

An item in a list can be accessed using the list index, starting from 0.
$Item = L [ \textless index\textgreater]$, e.g., $Item = L [0]$.

\iffalse
\item [Dictionary:]
The dictionary type is an associative array that holds
a pair of items, called a key-value pair. Keys in the dictionary have to be unique.

Dictionaries are created as:
$Dict = \{ \textless key \textgreater :  \textless val\textgreater,  \textless key \textgreater : \textless val \textgreater, ...\}$, e.g.,
$Dict = \{ 'Bob' : '1974', 'alice' : '1987'\}$.

A value in a dictionary can be accessed using its corresponding key.
$Val = Dict [ \textless key\textgreater]$, e.g., $Val = Dict ['Bob']$.
\fi
\end{description}

\subsection*{Dependencies}
Dependencies between programs can be specified using single arrows $\longrightarrow$ ,
whereas bidirectional arrows $\longleftrightarrow$ indicate that programs should be run in parallel.
This provides a straightforward job dependency expression, in the spirit of Python's
list comprehensions. We give some indicative examples:
%For grouping parentheses can be used.
\begin{lstlisting}
Example 1: Job B depends on Job A: 
program_a --> program_b

Example 2: Jobs A, B, C must run in parallel: 
program_a <-->  program_b <--> program_c

Example 3: Jobs A, B parallel, Job C dependent: 
(program_a <--> program_b ) <--> program_c
\end{lstlisting}

\subsection*{Functions and lambdas}
Functions can return a value and an error, and are created with the func
keyword. Lambdas are supported if no function name is given.
\begin{lstlisting}
func name(type arg1, type arg2) (return_type name, error_type err)
{
    //code here
    name = "my return object"
    err = nil  # or the error
    return
}
\end{lstlisting}

\subsection*{Conditional Statements and Errors}
\begin{description}
\item [Error checking:] If accepts a pre-statement that is executed (like the first
statement of a for loop)

Example:
$if (\_, err~=~execute\_job(j1);~ err~ !=~ nil) {}$
\end{description}


%
%if {}
%else\_if {}
%else{}
%
%if {}
%else{}
%
%
\subsection*{Loops}
\begin{description}
\item [For loop:] For loop is designed to iterate a fixed number of times.
For loop is declared as:

For $\textless var \textgreater~in~[\textless initialization \textgreater : \textless increment \textgreater: \textless termination \textgreater]~\{\textless statements \textgreater\}$

%Example:
%$For i IN [ 1 : 2 : 10] loops as 1, 3, 5, 7 and 9.$

\item [While loop:] while loop simply repeats the $\textless  statements\textgreater$ as long
as the $\textless  condition\textgreater$ is true. While loop is declared as:
$While (\textless  condition\textgreater)~\{\textless  statement\textgreater\}$

%Example:
%$int x = 5
%while ( x \textless 10 )
%{
%    print "Hello World"
%    x++
%}$

%This program prints Hello World as long as the value of the variable x is less than 10. Therefore, it loops 5 times.

\item [Do-While loop:] Do-While loop behaves like a While loop, except that $\textless condition\textgreater$
is evaluated after the execution of $\textless  statement\textgreater$ instead before.
%Guarantee 
%at least one execution of $\textless statement \textgreater$, even if $\textless condition \textgreater$ is never true. 
Do-While loop is declared as:
$Do\{\textless  statement\textgreater\}~While(\textless condition\textgreater)$
\end{description}


\subsection*{Comments}
Comments are also handled in a pythonic way, using a single pound sign '\#' for
one-line comments, and double pound signs '\#\#' to indicate start and end of block comments.

%Example:
%\# this is a comment
%
%\#\#
%	this is a block comment\\
%	this is a block comment
%\#\#


