\section{Language Specifications and Main Features - Arun, Ren}
\label{sect:spec}

\subsection{Operators}
perators essentially match those provided in python:

\begin{center}
    \begin{tabular}{| l | l | l |}
    \hline
    Arithmetic Operator & Description & Example where a = 4, b = 2
    + & Adds value on both sides of operator & a + b gives 6 \\ \hline
    - & Subtracts value to the right of the operator from value on left side of operator & a - b gives 2 \\ \hline
    / & Divides value on left side of operator by value on right side of operator & a/b gives gives 2 \\ \hline
    * & Multiplies values on both sides of operator & a*b gives 8 \\ \hline
    \% & Divides value on left side of operator by value on right side of operator and returns remainder & a \% b gives 0 \\ \hline
    
    \end{tabular}
\end{center}

Arithmetic operators respects traditional order of operations, that is multiplication and division 

+, -, /, \%, *, >, <, >=, <=, <>, +=, -=, ==, and, or, not, ->, <->

\subsection{Lists and Dictionaries}

- Lists

The list type is a container that holds a number of other objects, in a given order.
The list type implements a sequence, and also allows you to add and remove objects from the sequence.

Lists are created as:

L = [<expr>, <expr>, ...]
L = [23, 45, 2]

An item in a list can be accessed using the list index. Indices start from 0.

Item = L [<index>]
Item = L [0]



- Dictionary

The dictionary type is an associative array that holds a pair of items called a key-value pair.
Keys in the dictionary have to be unique.

Dictionaries are created as:

Dict = {<expr> : <expr>, <expr>, <expr>, ...}
Dict = { 'Bob' : '1974', 'alice' : '1987', ...}

A value in a dictionary can be accessed using its corresponding key.

Val = Dict [<key>]
Val = Dict ['Bob']




\subsection{Datatypes}

All variables have to be declared before they can be used. The datatype of the variable has to be specifies at declaration time.


- INT

Integer datatypes are declared using the 'Int' keyword. Integer variables can store whole number values like 0, 7, -1024
Variable names have to begin with an alphabet. Int support both signed and unsigned values.

Integer variables are declared as:

Int <var>, <var>, ...
Int <var> = <value>

Int a, b, c
Int a = 234


- FLOAT

Floating point datatypes can store real values like 1.03, -23.56. Floating Point variables are declared using the 'Float' keyword. Variable names have to begin with an alphabet.

Float variables are declared as:

Float <var>, <var>, ...
Float <var> = <value>

Float a, b, c
Float a = 23.78


- BOOL

Boolean datatypes can only hold one of two possible values: True or False. Boolean variables are declared using the 'Bool' keyword. Variable names have to begin with an alphabet.

Boolean variables are declared as :

Bool <var>, <var>, ...
Bool <var> = [<True> OR <False>]

Bool a, b, c
Bool a = True



- STRING

String datatypes represent a sequence of characters. String type variables are declared using the 'String' keyword. Variable names have to begin with an alphabet. String values are declared within single quotation marks.

String variables are declared as:

String <var>, <var>, ...
String <var> = ' <value> '

String a, b, c
String a = 'test'


\subsection{Dependences}
Maybe one-two example here? (bidirectional and single arrow?)

\subsection{Conditional Statments and Errors - Mathias}

\subsection{Function definition and lambdas - Mathis}

if {}
else\_if {}
else{}

if {}
else{}


\subsection{Loops}
for i in [1:2:10]{}

while(){}

do{}
while()



\subsection{Comments}
one hash for single line
two hashes for block comments


\label{sect:core}
