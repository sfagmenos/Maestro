\section{Language Specifications and Main Features - Arun, Ren}
\label{sect:spec}

\subsection{Operators}
Operators essentially match those provided in python:

\begin{center}
    \begin{tabular}{| l | l | l |}
    \hline
    Arithmetic Operator & Description & Example where a = 4, b = 2
    + & Adds value on both sides of operator & a + b gives 6 \\ \hline
    - & Subtracts value to the right of the operator from value on left side of operator & a - b gives 2 \\ \hline
    / & Divides value on left side of operator by value on right side of operator & a/b gives gives 2 \\ \hline
    * & Multiplies values on both sides of operator & a*b gives 8 \\ \hline
    \% & Divides value on left side of operator by value on right side of operator and returns remainder & a \% b gives 0 \\ \hline
    
    \end{tabular}
\end{center}

\begin{center}
    \begin{tabular}{| l | l | l |}
    \hline
    Comparison Operator & Description & Example where a = 4, b = 2 \\ \hline
    == & If values on both sides of operand are equal, condition becomes true & a == b is not true \\ \hline
    != & If values on both sides of operand are not equal, condition becomes true & a != b is true \\ \hline
    < & If value on left side of operand is strictly less than value on right side, condition becomes true \\ a < b is false \\ \hline
    > & If value on left side of operand is strictly greater than value on right side, condition becomes true & a > b is true \\ \hline
    <= & If value on left side of operand is less than or equal to value on right side, condition becomes true \\ a <= b is not true \hline
    >= & If value on right side of operand is greater than or equal to value on right side, condition becomes true & a >= b is true \\ \hline
    \end{tabular}
\end{center}

\begin{center}
    \begin{tabular}{| l | l | l |}
    \hline
    Assignment Operator & Description & Example where a = 4, b = 2 \\ \hline
    = & Assigns values from right side of the operator to left side operand & a = b means a has taken on the value of 2 \\ \hline
    += & Adds right operand to left operand and assigns result to left operand & a += b is equivalent to a = a + b\\ \hline
    -= & Subtracts right operand from left operand and assigns result to left operand \\ a -= b is equivalent to a = a - b \\ \hline
    += * Multiplies right operand and left operand and assigns result to left operand & a *= b is equivalent to a = a * b\\ \hline
    /= * Divides left operand by right operand and assigns result to left operand & a /= b is equivalent to a = a / b\\ \hline
    \end{tabular}
\end{center}

\begin{center}
    \begin{tabular}{| l | l | l |}
    \hline
    Logical Operator & Description & Example where a = true, b = false \\ \hline
    and & if both operands are true, then condition becomes true & a and b is false \\ \hline
    or & if either operand is true, then condition becomes true & a or b is true\\ \hline
    not & if condition is true, this operator will make it false. \\ not b is true \\ \hline
    \end{tabular}
\end{center}

\begin{center}
    \begin{tabular}{| l | l | l |}
    \hline
   	Membership operator & Description & Example where a = 'sam', b = ['sam', 'james', 'robert'] \\ \hline
    in & true if variable in specified sequence, false otherwise &  \\ a in b is true \hline
    not in & true if variable is absent from specified sequence, false otherwise & a not in b is false\\ \hline
    \end{tabular}
\end{center}

Order of operations is as follows 
\begin{enumerate}
\item \\ Exponentiation
\item \\ Multiply, divide, modulo
\item \\ Addition, Subtraction
\item \\ Comparison
\item \\ Assignment
\item \\ Membership
\item \\ Logical

+, -, /, \%, *, >, <, >=, <=, <>, +=, -=, ==, and, or, not, ->, <->

\subsection{Lists and Dictionaries}

- Lists

The list type is a container that holds a number of other objects, in a given order.
The list type implements a sequence, and also allows you to add and remove objects from the sequence.

Lists are created as:

L = [<expr>, <expr>, ...]
L = [23, 45, 2]

An item in a list can be accessed using the list index. Indices start from 0.

Item = L [<index>]
Item = L [0]



- Dictionary

The dictionary type is an associative array that holds a pair of items called a key-value pair.
Keys in the dictionary have to be unique.

Dictionaries are created as:

Dict = {<expr> : <expr>, <expr>, <expr>, ...}
Dict = { 'Bob' : '1974', 'alice' : '1987', ...}

A value in a dictionary can be accessed using its corresponding key.

Val = Dict [<key>]
Val = Dict ['Bob']




\subsection{Datatypes}

All variables have to be declared before they can be used. The datatype of the variable has to be specifies at declaration time.


- INT

Integer datatypes are declared using the 'Int' keyword. Integer variables can store whole number values like 0, 7, -1024
Variable names have to begin with an alphabet. Int support both signed and unsigned values.

Integer variables are declared as:

Int <var>, <var>, ...
Int <var> = <value>

Int a, b, c
Int a = 234


- FLOAT

Floating point datatypes can store real values like 1.03, -23.56. Floating Point variables are declared using the 'Float' keyword. Variable names have to begin with an alphabet.

Float variables are declared as:

Float <var>, <var>, ...
Float <var> = <value>

Float a, b, c
Float a = 23.78


- BOOL

Boolean datatypes can only hold one of two possible values: True or False. Boolean variables are declared using the 'Bool' keyword. Variable names have to begin with an alphabet.

Boolean variables are declared as :

Bool <var>, <var>, ...
Bool <var> = [<True> OR <False>]

Bool a, b, c
Bool a = True



- STRING

String datatypes represent a sequence of characters. String type variables are declared using the 'String' keyword. Variable names have to begin with an alphabet. String values are declared within single quotation marks.

String variables are declared as:

String <var>, <var>, ...
String <var> = ' <value> '

String a, b, c
String a = 'test'


\subsection{Dependences}
Dependencies between programs can be specified in the language using single $-->$ arrows. Bidirectional $<-->$ arrows indicate that programs should be run in parallel. This provides a clean easy syntax for straightforward job dependencies, in the spirit of Python's list comprehensions. For grouping, use parentheses.

Example 1:
Job B takes as input a file that Job A outputs. 'program_a --> program_b'

Example 2: 
Jobs $A$, $B$, and $C$ should be run in parallel. 'program_a <--> program_b <--> program_c'

Example 3:
Jobs $A$ and $B$ to be run in parallel, with Job $C$ dependent. ('program_a'<-->'program_b')--> 'program_c'
\subsection{Conditional Statments and Errors - Mathias}

\subsection{Function definition and lambdas - Mathis}

if {}
else\_if {}
else{}

if {}
else{}


\subsection{Loops}
for i in [1:2:10]{}

while(){}

do{}
while()



\subsection{Comments}
Comments are also handled the pythonic way : using a single pound sign '#' for one-line comments and double pound signs '##' to indicate start and end of block comments'

Example:
# this is a comment

##
	this is a block comment
	this is a block comment
	this is a block comment
##


\label{sect:core}
