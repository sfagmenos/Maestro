\section{Language Specifications and Main Features - Arun, Ren}
\label{sect:spec}

\subsection{Operators}
In what follows we present tables demonstrating the operators supported by our 
language, along with some indicative example. 


\begin{table}[h]
\begin{center}
    \parbox{.45\linewidth}{
        \begin{tabular}{| l | l |}
        \hline
        Operator & $a=4, b=2$ \\ \hline
        $+$ &  $a + b$ gives 6 \\  \hline
        $-$ &  $a - b$ gives 2  \\ \hline
        $/$ &  $a / b$ gives gives 2 \\ \hline
        $*$ &  $a * b$ gives 8  \\ \hline
        $\%$ & $a \% b$ gives 0 \\ \hline
        \end{tabular}
        \caption{Arithmetic operators.}
    }
    \parbox{.45\linewidth}{
        \begin{tabular}{| l | l |}
        \hline
        Operator & $a=4, b=2$ \\ \hline
        $=$  &  $a = b$  $a$ has taken on the value of 2 \\ \hline
        $+=$ &  $a += b$ is equivalent to a = a + b\\ \hline
        $-=$ &  $a -= b$ is equivalent to a = a - b \\ \hline
        $+=$ &  $a *= b$ is equivalent to a = a * b\\ \hline
        $/=$ &  $a /= b$ is equivalent to a = a / b\\ \hline
        \end{tabular}
        \caption{Assignment operators.}
    }
\end{center}
\end{table}

\begin{table}[h]
\begin{center}
    \parbox{.4\linewidth}{
        \begin{tabular}{| l | l |}
        \hline
        Operator & $a=4, b=2$ \\ \hline
        $==$ & $a == b$ is not true \\ \hline
        $!=$ & $a != b$ is true \\ \hline
        $\textless$  & $a \textless b$ is false \\ \hline
        $\textgreater$  & $a \textgreater b$ is true \\ \hline
        $\textless=$ & $a \textless= b$ is not true \\ \hline
        $\textgreater=$ & $a \textgreater= b$ is true \\ \hline
        \end{tabular}
        \caption{Comparison operators.}
    }
    \parbox{.4\linewidth}{
        \begin{tabular}{| l | l |}
        \hline
        Operator & $a = true, b = false$ \\ \hline
        and & a and b is false \\ \hline
        or & a or b is true\\ \hline
        not & not b is true \\ \hline
        in & a in b is true \\ \hline
        not in & a not in b is false\\ \hline
        \end{tabular}
        \caption{Logical operators}
     }
\end{center}
\end{table}

The order of supported operations 
is: (1) Exponentiation; (2) Multiply, divide, modulo; (3) Addition, Subtraction; 
(4) Comparison; (5) Assignment; (6) Membership; (7) Logical.


\subsection{Lists and Dictionaries}
\begin{description}
\item [Lists:] List type is a container that holds a number of other objects
 in a given order. The list type implements a sequence, and also allows you 
 to add/remove objects from the sequence.

Lists are created as:
$L = [\textless expr \textgreater, \textless expr \textgreater, ...]$, e.g., $L = [23, 45, 2]$.

An item in a list can be accessed using the list index, starting from 0.
$Item = L [ \textless index\textgreater]$, e.g., $Item = L [0]$.



\item [Dictionary:] The dictionary type is an associative array that holds 
a pair of items called a key-value pair. Keys in the dictionary have to be unique.

Dictionaries are created as:
$Dict = \{ \textless key \textgreater :  \textless val\textgreater,  \textless key \textgreater : \textless val \textgreater, ...\}$, e.g., 
$Dict = \{ 'Bob' : '1974', 'alice' : '1987'\}$.


A value in a dictionary can be accessed using its corresponding key.
$Val = Dict [ \textless key\textgreater]$, e.g., $Val = Dict ['Bob']$.
\end{description}



\subsection{Datatypes}

All variables have to be declared before they can be used. The datatype of a 
variable has to be specifies at declaration time.
\begin{description}
\item [Int:] Integer datatypes are declared using the 'Int' keyword. Integer 
variables can store whole number values like 0, 7, -1024. Variable names have 
to begin with an alphabet. Int support both signed and unsigned values.
Integer variables are declared as:
$Int \textless var\textgreater, \textless var\textgreater, ...$ 
%$Int \textlessvar\textgreater = \textlessvalue\textgreater$

%$Int a, b, c$
%$Int a = 234$


\item [Float:] Floating point datatypes can store real values like 1.03 or -23.56. 
Floating Point variables are declared using the 'Float' keyword. Variable names 
have to begin with an alphabet. Float variables are declared as:
$Float \textless  var\textgreater, \textless  var\textgreater, ...$
%$Float \textlessvar\textgreater = \textlessvalue\textgreater$

%Float a, b, c
%Float a = 23.78


\item [Bool:] Boolean datatypes can only hold one of two possible values: 
True or False. Boolean variables are declared using the 'Bool' keyword. Variable 
names have to begin with an alphabet. Boolean variables are declared as:
$Bool \textless  var\textgreater, \textless  var\textgreater, ...$
%$Bool \textlessvar\textgreater = [\textlessTrue\textgreater OR \textlessFalse\textgreater]$
%
%$Bool a, b, c$
%$Bool a = True$



\item [String:] String datatypes represent a sequence of characters. String type 
variables are declared using the 'String' keyword. Variable names have to begin
with an alphabet. String values are declared within single quotation marks.
String variables are declared as:
$String \textless var\textgreater, \textless  var\textgreater, ...$
%$String \textlessvar\textgreater = ' \textlessvalue\textgreater '$

%$String a, b, c$
%$String a = 'test'$
\end{description}

\subsection{Dependences}
Dependencies between programs can be specified in the language using single arrows $\rightarrow$,
whereas bidirectional arrows $\leftrightarrow$ indicate that programs should be run in parallel. 
This provides a straightforward job dependency expression, in the spirit of Python's 
list comprehensions. We give some indiocative examples:
%For grouping parentheses can be used.

Example 1: Job B depends on Job A: $program\_a \rightarrow program\_b$

Example 2: Jobs $A$, $B$, $C$ must run in parallel: $program\_a 
\leftrightarrow program\_b \leftrightarrow program\_c$

Example 3:
Jobs $A$, $B$ parallel, Job $C$ dependent:
$(program\_a \leftrightarrow program\_b ) \leftrightarrow program\_c$

\subsection{Conditional Statments and Errors - Mathias}
\begin{description}
\item [Functions:] Functions can return a value and an error, 
and are created with the func keyword. Lambdas are supported if no function name is given.

Example:
$func~~~name(type arg1, type arg2)(return\_type name, error\_type err)\{ //code here\}$ 
%
%$\{  // code here
%        name = "bla"
%        err = nil
%        return
%\}$

\item [Error checking:] ``If'' accepts a pre statement that will be executed (like the first 
statement of a for loop)

Example:
$if (\_, err~=~execute\_job(j1);~ err~ !=~ nil) {}$
\end{description}


%
%if {}
%else\_if {}
%else{}
%
%if {}
%else{}
%
%
\subsection{Loops}
\begin{description}
\item [For loop:] The For loop is designed to iterate a fixed number of times. 
For loop is declared as:

For $\textless var \textgreater~in~[\textless initialization \textgreater : \textless increment \textgreater: \textless termination \textgreater]~\{\textless statements \textgreater\}$

%Example:
%$For i IN [ 1 : 2 : 10] loops as 1, 3, 5, 7 and 9.$

\item [While loop:] The while loop simply repeats the $\textless  statements\textgreater$ as long 
as the $\textless  condition\textgreater$ is true. While loop is declared as:
$While (\textless  condition\textgreater)~\{\textless  statement\textgreater\}$

%Example:
%$int x = 5
%while ( x \textless 10 )
%{
%    print "Hello World"
%    x++
%}$

%This program prints Hello World as long as the value of the variable x is less than 10. Therefore, it loops 5 times.

\item [do-while loop:] It behaves like a while-loop, except that $\textless condition\textgreater$ 
is evaluated after the execution of $\textless  statement\textgreater$ instead before. Guarantee 
at least one execution of $\textless statement \textgreater$, even if $\textless condition \textgreater$ is never true. 
do-while loop is declared as:
$do\{\textless  statement\textgreater\}~while(\textless condition\textgreater)$
\end{description}


\subsection{Comments}
Comments are also handled the pythonic way, using a single pound sign '\#' for 
one-line comments and double pound signs '\#\#' to indicate start and end of block comments'

%Example:
%\# this is a comment
%
%\#\#
%	this is a block comment\\
%	this is a block comment
%\#\#


\label{sect:core}
