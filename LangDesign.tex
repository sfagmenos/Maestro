\section{Language Category and Design - Mathias}
\label{sect:desg}
\begin{description}
\item[Declarative] The goal is to have a declarative language to run jobs
(scripts) in a distributed way. The programmer only has to register workers and
declare dependencies. The language then figures out how to run the jobs and in
what order.
\item[Callbacks & lambdas] \lang will also support before and after callbacks
with a more imperative model to express any logic needed. \lang will support
lambdas to make callbacks easier to write.
\item[Strongly typed] \lang is strongly typed so that errors can be check
before jobs are sent. On top of classic types it will support a job type.
\item[Garbage collected] Memory will be handled by the language with a simple
garbage collector. Users should have to manage their memory is such a high
level and domain specific language, where performance is not a critical issue.
\item[No exceptions] Error checking will be handled on the spot, no exceptions
will be thrown to avoid hard to debug situations from asynchronous callbacks.
\item[Program structure] Programs will be written a sequence of function
definitions and declarations stored in a file or written in the REPL command line
interface.
\end{description}
