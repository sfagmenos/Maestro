\section{Language Environment}
\label{sect:tech}
The language environment has three major characteristics:
\begin{description}
\item[Workers list] The user shouldn't have to manage his workers after
they are launched; he only needs is to call a very basic program
to register his workers. \lang{} will then transparently assign jobs to workers
and manage dependencies.
\item[Job queue] Under the hood, jobs are stored in a priority queue. When a
worker becomes available, \lang{} will search the job queue in order of highest
priority first until it finds a job with all dependencies fulfilled. This job
will be assigned to the available worker.
\item[Interpreter and REPL] \lang{} is interpreted and has a REPL to be able to
quickly try things out and iterate. For really basic job launching, users shouldn't
even need to open a text editor.
\end{description}
