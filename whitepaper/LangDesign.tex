\section{Language Category and Design}
\label{sect:desg}
\begin{description}
\item[Declarative] The goal is to have a declarative language to run jobs
(scripts) in a distributed way. The programmer only has to register workers and
declare job dependencies; the language will schedule and run the jobs.
\item[Callbacks \& lambdas] \lang{} supports before/after callbacks
with a more imperative model to express any logic needed, and
% \lang{} supports
lambdas to make callbacks easier to write.
\item[Strongly typed] \lang{} is strongly typed so that errors can be check
before jobs are sent. On top of classic types it supports a job type.
\item[Garbage collected] Memory is handled by the language with a simple
garbage collector. Users should not have to manage their memory in a high
level domain specific language, where performance is not critical.
\item[No exceptions] Error checking is handled on the spot, and no exceptions
will be thrown to avoid hard to debug situations from asynchronous callbacks.
\item[Two return values] Functions return a value and an error, to make error
checking easier and compact, without throwing exceptions.
\item[Program structure] Programs are written as sequence of function
definitions and declarations stored in a file, or a REPL command line
interface.
\end{description}
