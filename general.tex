\section{General Instructions}
\label{sect:general}

\subsection*{Objects and Functions in Maestro}
\begin{lstlisting}
 New jobs are declared as objects of the $Job$ class using the implemented $new()$ method. The $new()$ method requires as parameters;
 $script$ which is the script to be executed on the worker(s).
 $workers$ which an optional parameters to mention the number of workers to run the job on.
 \\

 For example:
 
 Job b = Job.new( script=”xRay . rb” , worker s=3)
 Job c = Job.new(”telesphorus.py”)

 These create new jobs $b$ and $c$ as instances of the $Job$ class using the $new()$ method.
 \\
\end{lstlisting} 


\begin{lstlisting}
 Jobs are executed using the $run()$ method. The $run()$ method takes as parameter instances of the job class.
 The dependencies between jobs have to be mentioned. Dependencies are mentioned using $-->$ or $<-->$ as described in the whitepaper.

 For example:

 run ( ( a <−−> b) −−> c )
\end{lstlisting} 

