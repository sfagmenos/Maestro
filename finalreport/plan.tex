The first important decision we had to make as soon as we formed
our team was what language we are going to implement in the context of
``Programming Languages and Translators'' and whether it will be Domain
Specific or not. Having decided that we are going to implementva Domain
Specific Language, we met at the second week of the semester and brainstormed
about different domains of interesting problems to provide a language for.
For each suggested language we wrote sample programs to evaluate which will be
the most fruitful to implement. At the third week of the semester, we concluded
that we will implement \lang{} language and assign roles to team members.

\section{Specifications}
The next step was to define core specifications of \lang{} language in order to
be able to build a small running kernel of it. Once we had a basic kernel up
and running, we met with our mentor Junde Huang to demonstrate our progress
and get his feedback. Afterwards, we wrote our Language Reference Manual and
Tutorial, and start working on implementation of core features.
%Once we had full understanding of the procedure involved in building a compiler
%we implement a small kernel of

\section{Meetings and Communication}
During the semester we tried to meet weekly at least once to evaluate progress
on our project. The main purpose of our meetings was to ensure the each team
member has a full understanding of what she/he is expected to do, and also has a
clear picture of what other members are responsible for. Also, we
sometimes scheduled extra meetings when deadlines were due (language whitepaper,
manual, and tutorial). Apart from meetings, when additional communication was
necessary we exchanged daily e-mails to sync during evolution stage.

\section{Development (tools)}
For our development the single most important tool was GIT, a distributed
version control system. Multiple GIT branches were created during the semester
to keep track of our project. They included branches for source code
subversion (e.g., tests, translation, and \lang{} backend) as well as
branches for written manuscripts (e.g., whitepaper, tutorial,  and LRM).
For more details refer to Section~\ref{sec:deve}.

\section{Testing}
To make sure that the entire workflow smooth and to avoid bugs we created a
custom testing suit, built in python. Our testing environment is thoroughly
described in Section~\ref{sec:test}.


\section{Responsibilities}
In Table~\ref{tab:resp} we summarize responsibilities of each team member.
\begin{table}[!h]
{
 \begin{center}
    \begin{tabular}{ | l || l |}
    \hline
    \textbf{Team Member} & \textbf{Responsibility} \\
    \hline
    \hline
    V. Atlidakis & Backend Redis-based distributed communication protocol.\\ \hline
    M. Lecuyer & ... \\ \hline
    G. Koloventzos & ... \\ \hline
    Y. Lu & ...  \\ \hline
    A. Swaminathan  &  ... \\ \hline
    \hline
    \end{tabular}
    \caption{\textbf{Responsibilities.}}
    \label{tab:resp}
 \end{center}
}
\end{table}

\section{Implementation Style Sheet}



\section{Project Milestones}
In Table~\ref{tab:milestones}  we summarize project milestones and the date
each was fulfilled.

\begin{table}[!h]
{
 \begin{center}
    \begin{tabular}{ | l || l |}
    \hline
    \textbf{Date} & \textbf{Milestone}\\
    \hline
    \hline
    Feb 24 &  Language whitepaper\\ \hline
    Mar 26 &  Language reference manual and tutorial\\ \hline
    Mar 30 &  Compiler Front End (lexer and parser)\\ \hline
    Mar 30 &  Basic language kernel (without semantic actions)\\ \hline
    Mar 30 &  Hello World (locally)\\ \hline
    Apr 15 &  Initial test suite \\ \hline
    May 7 &  Semantics \& type checking complete \\ \hline
    May 8 &  Additional language features \\ \hline
    May 8 &  Backend redis-based distributed communication protocol.\\ \hline
    May 10 &  Final report\\ \hline
    May 11 &  Presentation\\ \hline
    \hline
    \end{tabular}
    \caption{\textbf{Project milestones.}}
    \label{tab:milestones}
 \end{center}
}
\end{table}


\section{Project Log}
In Figure~\ref{fig:projectlog.pdf}, we present a small project log.
For a detail project log refer to Section~\ref{sect:appendix}.


