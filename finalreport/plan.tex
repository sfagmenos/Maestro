Our first important decision as a team was what language we were going to implement and whether it would be domain-specific. We took a look at the examples from previous years and found that the most compelling ones did not attempt to replace general purpose languages like C/C++, but rather addressed quotidian concerns. So upon deciding on a domain-specific language, we met in the second week and began brainstorming problems that could be reasonably and competently solved in a class project. Some of our ideas included a probabilistic programming language that would make it easy to specify joint and conditional distributions, as well as a food-recipe language akin to What2Wear. We each wrote sample programs, did research, and presented our ideas. At the third week, we decided that we would implement \lang{} and assigned roles to each team member. 

\section{Specifications}
Our next step was defining core specifications of the \lang{} and getting a small kernel up and running. This eventually occurred sometime between the whitepaper and spring break, at which point we met with our mentor, Junde Huang. Junde had some good advice for organizing work flow. He also proposed adding a graphics unit to the language (which was ambitious and cool but did not eventually happen).

After spring break, we wrote our language reference manual and tutorial and started working on an implementation of additional features.

\section{Meetings and Communication}
During the semester we tried to meet weekly. The main purpose of our meetings was to ensure the each team member had a full understanding of what she/he was expected to do, and also full understanding of what other members were responsible for. Deadlines on deliverables called for extra scheduled meetings, and we also communicated through emails.

\section{Development}
For our development the single most important tool was GIT, a distributed
version control system. Multiple GIT branches were created during the semester
to keep track of our project. They included branches for source code
subversion (e.g., tests, translation, and the \lang{} backend) as well as
branches for written manuscripts (e.g., whitepaper, tutorial,  and LRM).
For more details refer to Section~\ref{sec:deve}.

\section{Implementation Style Sheet}
A five member team is relatively easy to manage and integrate relative to corporate settings, yet we discovered that good abstraction of responsibilities is essential: this allows the members of the team to work concurrently and make progress independently as the development evolved.  We did not set any restrictions on text editors or operating systems, but did ask of each other readable, commented python code. We also asked that partial commits of
code snippets come with meaningful commit messages. The source code and all written manuscripts of our project live under the private Columbia github repository.

\section{Testing}
To make sure that the entire workflow was smooth and to avoid bugs, we created a
custom testing suit, built in python. Our testing environment is thoroughly
described in Section~\ref{sec:test}.

\section{Responsibilities}
In Table~\ref{tab:resp} we summarize responsibilities of each team member.
\begin{table}[!h]
{%\small
 \begin{center}
    \begin{tabular}{ | l || l |}
    \hline
    \textbf{Team Member} & \textbf{Responsibility} \\
    \hline
    \hline
    V. Atlidakis & Backend Redis-based distributed communication protocol.\\ \hline
    M. Lecuyer & Lexical \& Syntax analysis, SDD to produce the AST, and translator. \\ \hline
    G. Koloventzos & Semantic analysis. \\ \hline
    Y. Lu & Test suite and language syntax \\ \hline
    A. Swaminathan  & Test suite and plan \\ \hline
    \hline
    \end{tabular}
    \caption{\textbf{Responsibilities.}}
    \label{tab:resp}
 \end{center}
}
\end{table}

\section{Project Milestones}
In Table~\ref{tab:milestones}  we summarize project milestones and the date
each was fulfilled.

\begin{table}[!h]
{%\small
 \begin{center}
    \begin{tabular}{ | l || l |}
    \hline
    \textbf{Date} & \textbf{Milestone}\\
    \hline
    \hline
    Feb 24 &  Language whitepaper\\ \hline
    Mar 17 &  Basic Compiler Front End (lexer and parser)\\ \hline
    Mar 26 &  Language reference manual and tutorial\\ \hline
    Mar 30 &  Basic language kernel (without semantic actions)\\ \hline
    Mar 30 &  Hello World (locally)\\ \hline
    Apr 15 &  Initial test suite \\ \hline
    May 7 &  Semantics \& type checking complete \\ \hline
    May 8 &  Additional language features \\ \hline
    May 8 &  Backend redis-based distributed communication protocol.\\ \hline
    May 10 &  Final report\\ \hline
    May 11 &  Presentation\\ \hline
    \end{tabular}
    \caption{\textbf{Project milestones.}}
    \label{tab:milestones}
 \end{center}
}
\end{table}


\section{Project Log}
In Section~\ref{sect:appen} we present a sample snapshot of some recent git
commits at the maestro branch that was used as our master source code branch.
