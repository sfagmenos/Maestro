\section{Development}
\label{sec:deve}
Our interpreter was developed in a Unix-based environment, specifically Debian, Ubuntu and MacOS.
Each member used their favorite text editor, including Vim, Sublime, and Emacs.  We used Github to host our
repository. As version control system we used Git. Some of us used tig for terminal access to git in
text mode. We did not used any issue tracking tools as 
our main development was done in sync with all members in the same room.
Our presentation was created, edited and distributed with Google Drive.


We used Python 2.7.3 and the Python-Lex-Yacc (PLY) to create and test our language. A generic Makefile
was created for the purpose of running all tests at once. Most of our examples are bash scripts.
For more sophisticated scripts such as map-reduce, we used ruby.


Our interpreter can take a file as argument. A REPL is started if no argument
is specified by the user.

\section{Runtime Environment}
One of the main goals of maestro was to distribute jobs and test them in computers with different
architectures and operating systems, so we wanted it to have as few environment dependencies as possible. Towards that end, our interpreter has 2 running options. 
For distributed functionality, there are a couple of configuration issues to first take care of: firstly, our interpreter needs a PC that can connect to the internet and can spawn threads. 
Secondly, download the following packages from Python.
\textit{ 
\begin{itemize}
\item pip install redis
\item pip install colorama
\end{itemize}
}
Our user does not require an active redis server. An online machine with the redis server installed will be used as a master channel. The user only needs to retrieve the ip of the machine and the redis port that is listening. 
\subsection{Local}
When run locally our interpreter creates processes to run jobs.
No changes are needed in the environment.
\subsection{Distributed}
For a distributed run the user must launch a master and a worker in redis in localhost. To create a master the user can specify \textit{service(IP:PORT)}. and for a worker he can specify \textit{worker(IP:PORT)}. Both of these functions are handled by our interpreter.
