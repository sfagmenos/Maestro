\section{Development}
Our interpreter was developed under a Unix-based environment, specifically Debian, Ubuntu and MacOS.
Every member used their favorite text editor. Mainly vim and sublime. We used Github to host our
repository. As version control system we used Git. Also some of us used tig for terminal access git in
text mode. We did not used any issue tracking tool as 
our main development was done synchronous with all members being in the same room.

We used Python 2.7.3 and the Python-Lex-Yacc (PLY) to create and test our language. A generic Makefile
was created for the purpose of running all tests at once. Most of our examples are bash scripts.
For more sophisticated scripts such as map-reduce, we used ruby.

Our interpreter can take a file as argument. Also a REPL is started if no argument
is specified by the user.

\section{Runtime Environment}
Our interpreter has 2 running options. Both of them does not require any change in user environment.
Our interpreter needs is a pc that can connect to the internet and the ability to spawn threads. For 
distributed run also needs a redis server for publishing messages with a master and a worker. Both of them can be also 
be done in localhost.
For creating a master user must specify \textit{service(IP:PORT)}. For worker he must
specify \textit{worker(IP:PORT)}.
