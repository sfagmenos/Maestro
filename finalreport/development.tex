\label{sect:development}
\section{Development}
Our interpreter was developed under a Unix-based environment, specifically Debian, Ubuntu and MacOS.
Every member used their favorite text editor. Mainly vim and sublime. We used Github to host our
repository. As version control system we used Git. Also some of us used tig for terminal access git in
text mode. We did not used any issue tracking tool as 
our main development was done synchronous with all members being in the same room.
Our presentation was created, edited and distributed with Google Drive.


We used Python 2.7.3 and the Python-Lex-Yacc (PLY) to create and test our language. A generic Makefile
was created for the purpose of running all tests at once. Most of our examples are bash scripts.
For more sophisticated scripts such as map-reduce, we used ruby.


Our interpreter can take a file as argument. Also a REPL is started if no argument
is specified by the user.

\section{Runtime Environment}
One of our main goals of maestro is to distribute jobs and test them in computers with different
architecture, different operating systems and many more. Because of this our environment should have
a small number or even none environment dependencies. In favor of this our interpreter has 2 running options. 
At the end small changes are needed for the distributed functionality.
What our interpreter needs is a pc that can connect to the internet and the ability to spawn threads. 
Many packages from python are needed such as
\textit{ 
\begin{itemize}
\item pip install redis
\item pip install colorama
\end{itemize}
}
Our user does not need to have a redis server running. A machine on the internet that has redis server
will be used as a master(channel). User only wants the ip of the machine and the port redis is listening.
Otherwise he can run it locally with no environment change.
\subsection{Local}
When run locally our interpreter is creating processes to run jobs.
No changes needed in the environment.
\subsection{Distributed}
For distributed run also needs a redis server for publishing messages with a master and a worker. 
Both of them can be also be done in localhost.
For creating a master user must specify \textit{service(IP:PORT)}. For worker he must
specify \textit{worker(IP:PORT)}. Both of these functions are handled by our interpreter
to connect to channel to publish jobs(service) or start a worker to consume jobs(worker).
