%\section{Introduction}
%\label{sect:wintro}

Frequently a user allocates a pool of machines in order to conduct experimental
evaluation represented in the format of scripts. In a broad sense, an end-user
operates on a pool of machines and allocates jobs represented in a scripting
language.
Our purpose is to build a declerative programming language, named \lang{},
to provide users with the ability to express complex semantics for job
distribution
and scheduling.
There are many scenarios where semantics are needed to express dependencies for
job execution, such as:
\begin{description}
\item[Jobs can depend on each other:] User needs to create and run jobs $A$,
$B$ in parallel, on a few workers. Only if both $A$ and $B$ succeed, job $C$
is allowed to execute.
%And then you want to run job $C$ on some
%workers, only after both $A$ and $B$ succeeded.
\item[Jobs can depend on external factors:] Job $A$ should execute after 6.00 PM
and job $B$ should execute in machines with CPU-load less than 0.5.
\end{description}

Prior related work in this area includes the so-called infrastructure
configuration management frameworks, some of the most popular being
Puppet~\cite{puppet} and Chef~\cite{chef}. Typical infrastructure
configuration management allows system administrators to express
infrastructure configuration dependencies. Our approach differs in that
it provides end-users the ability to express dependencies related to
job work-flows.

The rest of our document is structured as follows. In section~\ref{sect:desg} we
present the design of our language, in section~\ref{sect:spec} we underline
specifications and main features of our language. In section~\ref{sect:tech} we
describe the environment of our language, and then give an overview of what
programs will look like in section~\ref{sect:ex}. In section~\ref{sect:conc} we
give our concluding remarks.


\section{Language Category and Design}
\label{sect:desg}
\begin{description}
\item[Declarative] The goal is to have a declarative language to run jobs
(scripts) in a distributed way. The programmer only has to register workers and
declare job dependencies; the language will schedule and run the jobs.
\item[Callbacks \& lambdas] \lang{} supports before/after callbacks
with a more imperative model to express any logic needed, and
% \lang{} supports
lambdas to make callbacks easier to write.
\item[Strongly typed] \lang{} is strongly typed so that errors can be checked
before jobs are sent. On top of the classic types it supports a job type.
\item[Garbage collected] Memory is handled by the language with a simple
garbage collector. Users should not have to manage their memory in a high
level domain specific language, where performance is not critical.
\item[No exceptions] Error checking is handled on the spot. No exceptions
will be thrown to avoid hard to debug situations from asynchronous callbacks.
\item[Two return values] Functions return a value and an error, to make error
checking easier and more compact, without throwing exceptions.
\item[Program structure] Programs are written as a sequence of function
definitions and declarations stored in a file or a REPL command line
interface.
\end{description}

\section{Language Specifications and Main Features}
\label{sect:spec}

\subsection*{Operators}
In what follows we present tables demonstrating the operators supported by our
language.%, along with some indicative example.


\begin{table}[h]
\begin{center}
    \parbox{.45\linewidth}{
        \begin{tabular}{| l | l |}
        \hline
        Operator & $a=4, b=2$ \\ \hline
        $+$ &  $a + b$ gives 6 \\  \hline
        $-$ &  $a - b$ gives 2  \\ \hline
        $/$ &  $a~/~b$ gives gives 2 \\ \hline
        $*$ &  $a~*~b$ gives 8  \\ \hline
        $\%$ & $a~\%~b$ gives 0 \\ \hline
        \end{tabular}
        \caption{Arithmetic operators.}
    }
    \parbox{.45\linewidth}{
        \begin{tabular}{| l | l |}
        \hline
        Operator & $a=4, b=2$ \\ \hline
        $=$  &  $a = b$  $a$ has taken on the value of 2 \\ \hline
        $+=$ &  $a += b$ is equivalent to a = a + b\\ \hline
        $-=$ &  $a -= b$ is equivalent to a = a - b \\ \hline
        $+=$ &  $a *= b$ is equivalent to a = a * b\\ \hline
        $/=$ &  $a /= b$ is equivalent to a = a / b\\ \hline
        \end{tabular}
        \caption{Assignment operators.}
    }
\end{center}
\end{table}

\begin{table}[h]
\begin{center}
    \parbox{.4\linewidth}{
        \begin{tabular}{| l | l |}
        \hline
        Operator & $a=4, b=2$ \\ \hline
        $==$ & $a == b$ is not true \\ \hline
        $!=$ & $a != b$ is true \\ \hline
        $\textless$  & $a \textless b$ is false \\ \hline
        $\textgreater$  & $a \textgreater b$ is true \\ \hline
        $\textless=$ & $a \textless= b$ is not true \\ \hline
        $\textgreater=$ & $a \textgreater= b$ is true \\ \hline
        \end{tabular}
        \caption{Comparison operators.}
    }
    \parbox{.4\linewidth}{
        \begin{tabular}{| l | l |}
        \hline
        Operator & $a = true, b = false$ \\ \hline
        and & a and b is false \\ \hline
        or & a or b is true\\ \hline
        not & not b is true \\ \hline
        in & a in b is true \\ \hline
        not in & a not in b is false\\ \hline
        \end{tabular}
        \caption{Logical operators}
     }
\end{center}
\end{table}

The order of supported operations
is: (1) Exponentiation, (2) Multiplication, Division, Modulo, (3) Addition,
Subtraction, (4) Comparison, (5) Assignment, (6) Membership, and (7) Logical.

\subsection*{Datatypes}
Variables have to be declared before use, and datatypes of
variables are specified at declaration time. Variable names 
begin with an alphabetic letter.

\begin{description}
\item [Job:] The Job type is a \lang{} specific type, equivalent to a struct
with fields describing a Job to be run. The fields are: ``script'' for the
name of the script implementing the job, ``workers'' for the number of workers
to use (default is as many as possible), and ``priority'', an Int from 0 to 5,
0 being the
highest priority.
Job variables are declared as:
$Job~\textless var\textgreater = Job.new(~script=<String>,~workers=<Int>,~priority=<Int>)$

\item [Int:] Integer variables are used to store integer values like 0, 7,
-1024, and support both signed and unsigned values. Integer variables are
declared using the ``Int'' keyword as:
$Int~\textless var\textgreater,~\textless var\textgreater, ...$
%$Int \textlessvar\textgreater = \textlessvalue\textgreater$

%$Int a, b, c$
%$Int a = 234$

\item [Float:] Floating point datatypes are used to store real values like
1.03 or -23.56. Floating Point variables are declared using the ``Float''
keyword as:
$Float~\textless  var\textgreater,~\textless  var\textgreater, ...$
%$Float \textlessvar\textgreater = \textlessvalue\textgreater$

%Float a, b, c
%Float a = 23.78


\item [Bool:] Boolean datatypes are used to store one of two possible values:
True or False. Boolean variables are declared using the ``Bool'' keyword as
$Bool~\textless  var\textgreater,~\textless  var\textgreater, ...$
%$Bool \textlessvar\textgreater = [\textlessTrue\textgreater OR \textlessFalse\textgreater]$
%
%$Bool a, b, c$
%$Bool a = True$



\item [String:] String datatypes represent a sequence of characters. String
variables are declared using the ``String'' keyword, and string values are 
declared within single quotation marks.
String variables are declared as:
$String~\textless var\textgreater,~\textless  var\textgreater, ...$
%$String \textlessvar\textgreater = ' \textlessvalue\textgreater '$

%$String a, b, c$
%$String a = 'test'$

\item [List:]
List type is a container that holds objects of the same type in a given order.
The list type implements a sequence, and also allows the user to add/remove
objects from this sequence.

Lists are created as:
$L = type[\textless expr \textgreater,~\textless expr \textgreater, ...]$
, e.g., $L = int[23,~45,~2]$.

An item of a list can be accessed using the list index, starting from 0.
$Item = L [ \textless index\textgreater]$, e.g., $Item = L [0]$.

\iffalse
\item [Dictionary:]
The dictionary type is an associative array that holds
a pair of items, called a key-value pair. Keys in the dictionary have to be
unique.

Dictionaries are created as:
$Dict = \{ \textless key \textgreater :  \textless val\textgreater,  \textless key \textgreater : \textless val \textgreater, ...\}$, e.g.,
$Dict = \{ 'Bob' : '1974', 'alice' : '1987'\}$.

A value in a dictionary can be accessed using its corresponding key.
$Val = Dict [ \textless key\textgreater]$, e.g., $Val = Dict ['Bob']$.
\fi
\end{description}

\subsection*{Dependencies}
Dependencies between programs are specified using single arrows $\longrightarrow$ ,
whereas bidirectional arrows $\longleftrightarrow$ indicate that programs should be run in parallel.
This provides a straightforward job dependency expression, in the spirit of Python's
list comprehensions. We give some indicative examples:
%For grouping parentheses can be used.
\begin{lstlisting}
Example 1: Job B depends on Job A: 
program_a --> program_b

Example 2: Jobs A, B, C must run in parallel: 
program_a <-->  program_b <--> program_c

Example 3: Jobs A, B parallel, Job C dependent: 
(program_a <--> program_b ) <--> program_c
\end{lstlisting}

\subsection*{Functions and lambdas}
Functions return a value and an error, and are created with the func
keyword. Lambdas are supported if no function name is given.
\begin{lstlisting}
func name(type arg1, type arg2) (return_type name, error_type err)
{
    //code here
    name = "my return object"
    err = nil  # or the error
    return
}
\end{lstlisting}

\subsection*{Conditional Statements and Errors}
\begin{description}
\item [Error checking:] If accepts a pre-statement that is executed (like the
first statement of a for loop)

Example:
$if (\_, err~=~execute\_job(j1);~ err~ !=~ nil) {}$
\end{description}


%
%if {}
%else\_if {}
%else{}
%
%if {}
%else{}
%
%
\subsection*{Loops}
\begin{description}
\item [For loop:] For loop is designed to iterate a fixed number of times.
For loop is declared as:

For $\textless var \textgreater~in~[\textless initialization \textgreater : \textless increment \textgreater: \textless termination \textgreater]~\{\textless statements \textgreater\}$.

%Example:
%$For i IN [ 1 : 2 : 10] loops as 1, 3, 5, 7 and 9.$

\item [While loop:] while loop simply repeats the $\textless  statements\textgreater$ as long
as the $\textless  condition\textgreater$ is true. While loop is declared as:
$While (\textless  condition\textgreater)~\{\textless  statement\textgreater\}$.

%Example:
%$int x = 5
%while ( x \textless 10 )
%{
%    print "Hello World"
%    x++
%}$

%This program prints Hello World as long as the value of the variable x is less than 10. Therefore, it loops 5 times.

\item [Do-While loop:] Do-While loop behaves like a While loop, except that $\textless condition\textgreater$.
is evaluated after the execution of $\textless  statement\textgreater$ instead of before.
%Guarantee 
%at least one execution of $\textless statement \textgreater$, even if $\textless condition \textgreater$ is never true. 
Do-While loop is declared as:
$Do\{\textless  statement\textgreater\}~While(\textless condition\textgreater)$
\end{description}


\subsection*{Comments}
Comments are also handled in a pythonic way, using a single pound sign ``\#'' for
one-line comments, and double pound signs ``\#\#'' to indicate start and end of block comments.

%Example:
%\# this is a comment
%
%\#\#
%	this is a block comment\\
%	this is a block comment
%\#\#

\section{Language Environment}
\label{sect:tech}
The language environment has three major characteristics:
\begin{description}
\item[Workers list] The user shouldn't have to manage his workers after
they are launched; he only needs to call a very basic program
to register his workers. \lang{} will then transparently assign jobs to workers
and manage dependencies.
\item[Job queue] Under the hood, jobs are stored in a priority queue. When a
worker becomes available, \lang{} will search the job queue in order of highest
priority first until it finds a job with all dependencies fulfilled. This job
will be assigned to the available worker.
\item[Interpreter and REPL] \lang{} is interpreted and has a REPL to be able to
quickly try things out and iterate. For really basic job launching, users shouldn't
even need to open a text editor.
\end{description}

\section{Examples}
\label{sect:ex}
Example 1: Create and run 2 jobs $A$ and $B$ in parallel, on a few
workers. Then, you want to run job $C$ on some workers, only after both $A$
and $B$ succeeded.
\begin{lstlisting}
Job a = Job.new(script="abc.pl", workers=5)    /* job scripts */
Job b = Job.new(script="xRay.rb", workers=3)
Job c = Job.new("telesphorus.py")
run((a <--> b) --> c)
\end{lstlisting}
Example 2: Make a measurement that is divided into 3 steps: populate, collect
data, and analyze.
\begin{lstlisting}
func an_hour_after(Job j) {
  return func () {
    j.finish_time && j.finish_time - Time.now > Time.hour(1)
  }
}

Job old_b = nil
for i in range(0,10) {
  a = Job.new(script='send_email_batch.rb', workers=1)
  b = Job.new(script='collect_data.rb', workers=1)
  b.add_func_dependency( an_hour_after(a) )
  Job c = Job.new(script='analyze.py')
  run(old_b --> a --> b --> c)
  old_b = b
}
\end{lstlisting}

\section{Conclusion}
\label{sect:conc}
In this white paper we present an introductory description of \lang{}, a
language intended to help users describe and orchestrate jobs based on a-priory
known dependencies. In the future, during the evolution of our project, there
may be additional functionality implemented, but the core specifications of
\lang{} will remain as is.
