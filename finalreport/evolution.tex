\section{Look and Feel}
The goal of the language was clear from the very beginning - Maestro was designed to make it easy to schedule jobs, with dependencies between them, or in parallel, locally, or over the network.
With that in mind, we were mostly able to hem to the original specifications in the whitepaper.
In order to get an idea of the uses cases we had in mind and the semantics we wanted, we all wrote three to five programs in maestro.
Great ideas came out of this exercise, like our dependencies operators.

\section{Implementation and  Tools}
There were, however, some changes and feature additions. Originally, the team had decided to do the project in C. C became Python when it became clear that Maestro would be used primarily as a scripting language. We thus decided to implement an interpreter instead of a compiler.
We also chose Python because we wanted to use some of its higher level libraries to implement network operations, distributed map/reduce, and dynamic dependencies between Jobs with lambdas.
We thus decided to use PLY to implement Maestro's frontend.

Redis came into the picture when we started to distribute jobs on remote workers.
Fortunately it was compatible with the architecture, since it had always been on our "would be great to have" list.
We use the python Redis package to interface with the Pub/Sub feature of redis.

\section{Added Features}
The first big feature we changed is the way we do recursion in Maestro. We were planning classic for and while loops, but realized we were always looping over lists.
We thus changed the for loops to the each syntax. This is similar to how loops are handled in Ruby and is much more compact and enjoyable.

Another feature that we introduced late is map/reduce. When thinking about the use cases (after we had a basic "Hello World" working), most of them involved complex processing and dependencies, and where a bit verbose or even too hard to express because we needed to wait on other jobs' outputs.
We thus introduced a map/reduce abstraction, which allows us to keep our asynchronous model, but still use the output of previous jobs.
This is really adapted to remote job distribution and execution, and fits perfectly in maestro.

We also added overloading for + and map functions: the behavior change depending on the type of the arguments.
We didn't think about it beforehand, but when we hit the problem it was a natural extension.

A last unexpected improvement we added was nice console outputs of the job queue and state of dependencies.
The language reference manual and the tutorial were updated with the languages when we changed features.

\section{Dropped Features}

We are really happy we implemented the job distribution feature.There still are some features on our "would be great to have list":
We wanted to add more error handling features, functions and lambdas for dependency callbacks, and scoping for variables.
We believe our architecture can support it, but we chose to focus on rmote job execution.

