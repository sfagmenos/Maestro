\section{Methodology}
We do not rely on any existing Python unit-test framework for our testing purposes. Instead, we have developed a customized testing engine for our project. The engine, encapsulated by the Tests folder, comprises of several Python files. Each file consists of a test written in Maestro that checks both the robustness of the compiler (i.e. its ability to identify and parse tokens), and the final output of the program. 

Presented below is a sample unit test.

\begin{center}
\begin{verbatim}
#!maestro
master("0.0.0.0:6379");
# check what happens when a job is run without being declared
a = Job('print.py', 'foo');
c = Job('print.py' 'bar');
run(a -> b -> c);  # Job b is not declared
\end{verbatim}
\end{center}

The test first checks if two Jobs are created successfully. The file \textit{print.py} is a Python program that writes the argument passed to it to a file \textit{check.txt}. Once executed, Job \textit{a} must run the argument \textit{foo} through the program \textit{print.py}. Similarly, Job \textit{b} must run the argument \textit{bar} through \textit{print.py}. 
Once the two Jobs have been created successfully, we attempt to pass an undeclared Job to the \textit{run} method, which executes a single Job or a list of Jobs in the given order. In this case, we check to see if the Maestro compiler identifies undeclared Job \textit{b} as an error and handles it elegantly. Once the test completes execution, we check the expected output against the output that was piped to \textit{check.txt} to confirm success or failure. 

We have a total of 75 testcases in our engine that test Maestro's  overall performance in different steps.

\section{Relevant Files and Descriptions}

\textbf{testframework.py}
\newline
\indent This is the main framework file. It can execute all Maestro test-cases or individual ones, depending on how it is called. 
\newline \textit{python testframework.py all} executes all test cases in the \textit{Tests} folder. Replacing \textit{all} with the test-name will execute individual tests. 
\newline
\newline
\textbf{run.sh}
\newline
\indent Similar to testframework.py, but written in bash script.
\newline
\newline
\textbf{Tests\textbackslash print.py}
\newline
\indent This program is run by every Job created in our test cases. It writes the argument that it receives into file \textit{check.txt}. 
\newline
\newline
\textbf{Tests\textbackslash}
\newline
\indent This folder contains all 75 Maestro test cases.
\newpage
\noindent\textbf{TestsOutput\textbackslash check.txt}
\newline
\indent The checkfile to which \textit{print.py} writes its output. We compare the contents of the file against the expected output to determine success or failure of a given Job.
\newline
\newline

\section{Selected Test Cases}
\begin{enumerate}

\item This test checks the performance of Maestro in presence of a circular dependency. 
\begin{verbatim}
#!maestro
#master("0.0.0.0:6379")
# check hard circular dependencies with strings
a = Job("print.py", "cheer")
b = Job("print.py", "us")
c = Job("print.py" "on")
run(a -> b -> c -> a)
\end{verbatim}
Here job \textit{a} depends on \textit{b}, \textit{b} depends on \textit{c} and job \textit{c} in-turn depends on job \textit{a}.

\item This test checks the performance of Maestro on self dependencies.
\begin{verbatim}
#!maestro
#master("0.0.0.0:6379")
# check what happens when jobs depend on themselves
a = Job('print.py', 'foo')
run(a -> a -> a)
\end{verbatim}
\newpage
\noindent \item This test check the performance of Maestro on nested loops.
\begin{verbatim}
#!maestro
#master("0.0.0.0:6379")
# check use of the operator *
c = 5
d = 8
range(d).each(var){
	range(c).each(varr){
		a = Job('print.py', var*varr)
		run(a)
	}

}
\end{verbatim}

\item This test case checks what happens when we just run wait expressions in circular.
\begin{verbatim}
#!maestro
#master("0.0.0.0:6379")
# check what happens when we just run wait expressions in circular?
a = Job('print.py', 'cool')
b = Job('print.py', 'it')
c = Job('print.py', 'now')
run(wait(10) ~> wait(10) ~> c ~> wait(10))
\end{verbatim}

\end{enumerate}
