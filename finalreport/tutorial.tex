%\section{Introduction}
%\label{sect:intro}

%Frequently a user allocates a pool of machines in order to conduct experimental
%evaluation represented in the format of scripts. In a broad sense, an end-user
%operates on a pool of machines and allocates jobs represented in a scripting
%language. \lang{} is a programming language which provides user the ability
%to express powerful semantics related to job distribution and scheduling.
The purpose of this tutorial is to demonstrate features and functionality
supported by \lang{}. We, also,  provide exemplary programs to explain how
\lang{} should be used to create jobs, describe dependencies amongst jobs, and
execute jobs on workers. Our examples include a simple ``Hello World'' program as
well as more complex programs which cover \lang{}'s functionality in detail.


\section{General Instructions}
\label{sect:general}

\subsection*{Define Workers in \lang{}}
%Refine later
In the first version of \lang{}, we will use a Redis~\cite{redis} server to
implement. To add a worker to the workers' pool, the only thing one has to do
is run the following Maestro program: worker("0.0.0.0") (replace 0.0.0.0 by
the IP of the Redis server).\\
This will subscribe the worker to a job channel in Redis, where all Jobs will
be published for execution when a master runs the run command.\\
The master calls master("0.0.0.0") (replace 0.0.0.0 by the IP of the Redis server)
at the beginning of its program.

\subsection*{Variables and Expressions in \lang{}}
In \lang{} variables are reserved memory locations which store values.
Practically, when a user declares a variable memory space is allocated.

Based on the data type of a variable, the interpreter allocates memory
and decides what can be stored in the reserved memory. Therefore, by
assigning different data types to variables, user can store integers, Jobs,
or strings.

\subsubsection*{Assignment operator}
\lang{}  is a dynamically typed language and variables types are not explicitly declared.
Variable types are decided when a variable is assigned a value with the equal
assignment operator ``=''.

The operand on the left side of assignment operator ``='' must be a variable name
and the operand on the right side of assignment operator ``='' must be a value.
The variable on the left side of ``='' is assigned the value on the right side
of ``=''.\\
\\
Example:
\begin{Verbatim}[numbers=left]
a = 10;
x = "john";
path = "/afs/columbia.edu/users/phd/username/my_script.sh";
p = true;
\end{Verbatim}

The above code snippet demonstrates the following assignments:
In line one variable ``a'' is assigned integer value 10.
In line two variable ``x'' is assigned string ``john''.
In line three variable ``path'' is assigned a string representing the absolute
path of some script.
Finally, in line four variable ``p'' is assigned boolean value true.

\subsubsection*{Job}
The most important type of variable that can be created in \lang{}, is the
Job variable.\\
\\
Example:
\begin{Verbatim}[numbers=left]
a = Job ("abc.rb");
\end{Verbatim}
In this case variable ``a'' is assigned a job related to script ``abc.rb'' 

\subsection*{each}

\noindent ``each'' keyword is used to feed a list of variables as input
to a block of statements. Its syntax is:

\textit{[val1, val2, ..., valN].each(var)\{ statements\}}\\

The interpretation is that ``statements'' are executed as many times as the
number of variables in the list [val1, val2, ..., valN]. For the first
execution, var is assigned
the value val1 and is fed to the statements. For the second execution var is
assigned the value
of val2 and is fed to the statements. This process is repeated until every value 
from 1 to N has been assign to var.\\
\\
Example:
\begin{Verbatim}[numbers=left]
vals = ["tiny.txt", "medium.txt", "large.txt"];
vals.each(var){
    a = Job("/usr/bin/grep", "compiler", var);
    run(a);
}
\end{Verbatim}

In the above code snippet, Job ``a'' is instantiated and executed three times
with different arguments each time.



\subsection*{Operators in \lang{}}

\lang{} language supports left associative operators with special meaning:
operators $->$ and $<->$ for concurrency. 
Also operators $\sim$, $\sim<$ and $\sim>$ for job dependencies.
\subsubsection*{Asynchronous operator $(->)$}
To allow user to define that 2 processes must run one after another
we implement the asynchronous operator $(->)$. By using asynchronous operator,
the user specifies that the process at right of the arrow can start as soon as
the process on the left-hand side of the arrow terminates successfully.
\subsubsection*{Concurrent operator $(<->)$}
Unlike the asynchronous operator, concurrent operator $(<->)$ denotes that 2 jobs can
run in parallel. Failure of the one does not preclude the other.
\subsubsection*{Equal dependency operator ($\sim$)}
Equal dependency operator ($\sim$) states that two jobs may run concurrently if
there are enough resources.
\subsubsection*{Pre dependency operator ($\sim>$)}
Pre dependency operator ($\sim>$) states that the job on its left may run
before the job on its right.
\subsubsection*{Post dependency operator ($\sim<$)}
Post dependency operator ($\sim<$) states that the job on its left should run
after the job on its right.

\subsection*{Classes and Functions in \lang{}}
\subsubsection*{Class $Job(string <name>,~ string <script\_path>,~arg1,~arg2, ...,~argN)$}
Construction of a $Job$ type requires the following variables:
\begin{itemize}
\item $name$ which is the name of the job.
\item $script\_path$ which is the script to be executed.
\item A list of arguments related to script under $script\_path$.\\
\end{itemize}
For each \lang{} job a log file is created under the user's current working
directory. This
log file is named after $<script\_path>$ along with an additional ".log" suffix. The purpose
of a log file is to store diagnostic messages and output generated by the respective job script.

Example:
\begin{Verbatim}[numbers=left]
b = Job("xRay.rb","arg1", "arg2");
c = Job("telesphorus.py", "arg1");
\end{Verbatim}

The above snippet creates two jobs ``b'' and ``c'' as instances of class $Job$.
Job ``b'' is assigned the execution of script ``xRay.rb'' with arguments ``arg1'' and ``arg2''.
Job ``c'' is assigned the execution of script ``telesphorus.py'' with arguments ``arg1'' and ``arg2''.


\subsubsection*{\lang{} Functions}

This section we present core functions of \lang{}.
\subsubsection*{$run($\textit{expr}$)$}
\textit{run} function has one argument, which is an expression related to jobs.
\textit{expr} can be jobs:
\begin{Verbatim}[numbers=left]
a = Job(script="abc.pl", "10");
b = Job(script="xRay.rb", "mailbox");
c = Job("telesphorus.py", "android_logs");
run(a,b,c);
\end{Verbatim}

or a ``hard'' dependency among jobs:
\begin{Verbatim}[numbers=left]
a = Job(script="abc.pl", "10");
b = Job(script="xRay.rb", "mailbox");
run(a->b);
\end{Verbatim}
\subsubsection*{$run\_local($\textit{expr}$)$}
This is a local version of the ``run'' function.
With ``run\_local'' function jobs may run locally for testing purposes before
user launches a program remotely with ``run''.


\section{Sample Programs}
\label{sect:samples}
In order to run the programs, we just need to call maestro program\_path. If the \lang{} 
program is not in the PATH, it need to be specified (e.g. /home/user/bin/maestro).\\
We can also start the REPL with maestro, and write the commands.

\subsection*{Example Program 1: Hello World}
Consider a Ruby script named $hello\_world.rb$ containing the following:
\begin{verbatim}
#!/usr/bin/env ruby
puts 'Hello World!'
\end{verbatim}
Now consider a \lang{} program like the following:
\begin{Verbatim}[numbers=left]
// Hello World
a = Job("hello_world.rb");
run(a);
\end{Verbatim}

In the above code snippet we demonstrate how to create of a job running a
hello world script.
This job is then initialized with the script of our hello world program and
assigned to variable `a'. In line 3, $run()$ starts job `a'. 
\lang{} will find an idle worker and execute our script hello\_world.rb.
Afterwards, error code of job `a' is evaluated. If script hello\_world.rb terminates properly
,its stdout will be loged on the master. Otherwise, stderr will be logged.

\subsection*{Example Program 2: Distributed Hello World}
Consider a Ruby script named $print.rb$ that prints its first argument:
\begin{verbatim}
#!/usr/bin/env ruby
puts ARGV.first
\end{verbatim}
We then start workers running this maestro program:
\begin{Verbatim}[numbers=left]
worker("0.0.0.0:6379");
\end{Verbatim}
The Maetro program can be:
\begin{Verbatim}[numbers=left]
master("0.0.0.0:6379");
a = Job("print.rb", "Hello");
b = Job("print.rb", "World");
c = Job("print.rb", "!");
run(a -> b -> Wait(10) -> c);
\end{Verbatim}

There is a lot going on in this example. It uses the concurrency capabilities
of \lang{} to express dependencies and
waiting times. We can also see how to run a Job with an argument.\\
First, we declare the IP:PORT of Redis (here localhost on the default Redis
port). It will be used to
distribute the Jobs.\\
3 jobs are created, and dependencies amongst them are specified.
Line 5 shows a simple example of how to introduce dependencies in \lang{}.
The expression $a -> b$ states that job `a' will run before `b', potentially
on different servers. Afterwards,  $-> c$ indicates that the job `c' must run
after the
successful termination of the previous jobs (`a' and `b').\\
The Wait() call creates a fake Job which only purpose it to delay the next Job.
The hard dependency system
ensures `c' will not run before 10 seconds after `b' finished.\\
If we wanted `a' and `b' to run in parallel, and add the ! at the end after 10
seconds, we could call:
\begin{Verbatim}[numbers=left]
run((a <-> b) -> Wait(10) -> c);
\end{Verbatim}

You can see that \lang{} offers an easy way to distinguish between
concurrent and non concurrent jobs. This will help the user create and easily
distribute batch processes.\\
\\

\subsection*{Example Program 3:}
\begin{Verbatim}[numbers=left,commandchars=\\\{\}]
master("0.0.0.0:6379");
a = Job("abc.pl");
b = Job("xRay.rb");
c = Job("telesphorus.py");
d = Job("nimbledriod.py");
((a\mytilde>(b\mytilde c))\mytilde<d);
run((a,b,c,d);
\end{Verbatim}

This example demonstrates {\em soft} dependencies in \lang{}.\\
`Soft' dependencies must be all in one line. Operators
are left associative but we use parentheses in order to help
the user understand how `soft' dependencies work. Line 5 is where all the
magic happens. First, $(b\sim c)$ means that jobs `b' and `c' can run 
synchronously if there are enough resources. Second,
$(a\sim>expr)$ means that job `a' should run before \textit{expr}. 
Finally, $(expr\sim<d)$ means that job `d' should run before \textit{expr}.\\
\\

\subsection*{Example Program 4:}
\begin{Verbatim}[numbers=left,commandchars=\\\{\}]
master("0.0.0.0:6379")
list = [];
range(100).each(i) \{
  a = Job("send_email.pl", "" + i, "test@maestro.com");
  list = list + a;
\}
b = Job("collect_data.rb");
list -> Wait(3600) -> b;
run(b);
\end{Verbatim}

This examples shows how to schedule many instances of the same Job with different
arguments, and wait for all of them to be done before starting another script.
In this example, send\_email takes an email number and an email address as arguments,
and sends the email to the address. The program creates one Job per email so that it
can be distributed. Then it waits for an hour after the last email, and starts some data
collection dependent on the emails. It assumes all workers have access to the email list
so that they can retrieve the right one (it could be hard coded in the script or through a distributed
file system).
Execution of job `b' will start only on success of all email sending,
and after an one-hour wait.

\subsection*{Example Program 5:}
\begin{Verbatim}[numbers=left,commandchars=\\\{\}]
master("0.0.0.0:6379");
a = Job("split.rb", "/tmp/big_file_name.data");
maps = map(a, "map.rb");
red = reduce(maps, "reduce.rb");
run(red);
\end{Verbatim}

This example shows how to use the map/reduce built in functions to distribute data
analysis of a big file. This assumes that all workers have access to the file (e.g.
distributed file system), or that split.rb can scp the pieces of the file in the workers.
Then map will create one Job per output line of split.rb. This Job will be the map.rb
script and have as argument one line of split output. Reduce will then create one Job
with the reduce.rb program, with its arguments being the output of the previous
reduce and the output of the next Job from the list. The first reduce will have
nil as first argument.\\
Let us assume the big file contains words, and we want to count the number of appearances for
each different word:
\begin{itemize}
  \item split.rb can split the big file in chunks and output the name of each chunk.
  \item map.rb can take a file path, and return a dictionary with the words as keys
    and an int for the value, in the JSON format.
  \item reduce.rb can take 2 arguments as JSON strings and parse them. If the first is
    empty, it means it is the initialization and we create an empty dictionary. Then it
    merges the dictionaries and returns the merged in JSON.
\end{itemize}


\section{Conclusion}
\label{sect:conclusion}
This language tutorial presents the core functionality of \lang{}. It includes examples
demonstrating how \lang{} primitives should be used to create jobs, specify dependencies, and run
jobs. For a complete analysis refer to our Language Reference Manual.
